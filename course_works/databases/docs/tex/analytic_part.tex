\chapter{Аналитическая часть}

В этой части рассматриваются анализ предметной области, известных решений и моделей баз данных, формализации задачи и ролей.

\section{Анализ предметной области}

Организация мероприятий -- это многогранный и трудоемкий процесс, который требует внимания к деталям и учета множества факторов. От выбора подходящей локации до составления списка гостей, от планирования бюджета до подготовки меню -- каждый этап организации требует тщательной проработки. В ходе подготовки организаторы сталкиваются с рядом типичных вопросов, таких как: «Что необходимо приобрести для мероприятия?», «Какое количество гостей ожидается?» и «Какова стоимость участия?». Эти вопросы, хотя и кажутся простыми, но требуют значительных временных и организационных затрат, особенно если мероприятие масштабное или включает множество участников~\cite{lit1}.

Для упрощения этого процесса были разработаны специализированные инструменты -- планировщики мероприятий. Эти приложения предназначены для того, чтобы объединить все этапы организации мероприятия в единую систему, сделать процесс планирования более структурированным и прозрачным. Планировщик мероприятий позволяет организаторам:
\begin{enumerate}
	\item систематизировать задачи -- разбить процесс организации на этапы и подзадачи;
	\item координировать участников -- вести список гостей, учитывать их предпочтения и информировать о деталях мероприятия;
	\item управлять бюджетом -- учитывать расходы и планировать финансы, чтобы избежать непредвиденных затрат;
	\item контролировать сроки -- устанавливать дедлайны для каждой задачи и отслеживать их выполнение.
\end{enumerate}


\section{Анализ известных решений}

\section{Формализация задачи}

\section{Формализация ролей}

\section{Анализ моделей баз данных}

\section{Вывод}

В аналитической части работы был проведен анализ...

\clearpage
