\ssr{ВВЕДЕНИЕ}

Компьютерная графика является одной из самых динамично развивающихся областей информационных технологий, играющей ключевую роль в современном обществе. Она охватывает широкий спектр приложений, включая игры, анимацию, визуализацию данных, дизайн, архитектуру и медицину.

Целью данной курсовой работы является разработка программы композиции и визуализации трехмерных многогранных примитивов с учётом их геометрических и оптических параметров, вводимых пользователем. Для удобства восприятия должны присутствовать эффект глубины визуализируемой сцены, опции перемещения камеры, настройка положения источника света и возможность изменения его спектральных характеристик.

Чтобы достичь данной цели, необходимо выполнить следующие задачи:
\begin{itemize}[label=--]
	\item описать объекты сцены;
	\item проанализировать существующие алгоритмы компьютерной графики для генерации реалистичных моделей и трехмерной сцены;
	\item выбрать наиболее подходящие алгоритмы для достижения поставленной цели;
	\item спроектировать архитектуру и графический интерфейс приложения;
	\item выбрать средства реализации программного обеспечения;
	\item реализовать выбранные алгоритмы и структуры данных;
	\item провести исследование быстродействия разработанного приложения.
\end{itemize}

\clearpage
