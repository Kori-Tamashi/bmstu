\addcontentsline{toc}{chapter}{СПИСОК ИСПОЛЬЗОВАННЫХ ИСТОЧНИКОВ}
\begin{thebibliography}{}
	\bibitem{lit1} Аксенов А. Ю. Модели и методы обработки и представления сложных пространственных объектов : специальность 05.13.01 «Системный анализ, управление и обработка  информации (технические системы)» : Автореферат на соискание кандидата технических наук / Аксенов А. Ю. ; СПИИРАН. — Санкт-Петербург, 2015. — 22 c.
	\bibitem{lit2} Colin Smith On Vertex-Vertex Meshes and Their Use in Geometric and Biological Modeling / Colin Smith [Электронный ресурс] // Algorithmic Botany : [сайт]. — URL: \url{http://algorithmicbotany.org/papers/smithco.dis2006.pdf} (дата обращения: 17.07.2024).
	\bibitem{lit3} Роджерс Д. Алгоритмические основы машинной графики: Пер. с англ. [Текст] / Роджерс Д. — Санкт-Петербург: Мир, 1989 — 512 c.
	\bibitem{lit4} Шикин Е. В., Боресков А. В. Компьютерная графика. Динамика, реалистические изображения [Текст] / Шикин Е. В., Боресков А. В. — Москва: ДИАЛОГ-МИФИ, 1996 — 288 c.
	\bibitem{lit5} Гамбетта Гэбриел. Компьютерная графика. Рейтрейсинг и растеризация [Текст] / Гамбетта Гэбриел. — Санкт-Петербург: Питер, 2022 — 224 c.
	\bibitem{lit6} Польский С. В. Компьютерная графика : учебн.-методич. пособие. [Текст] / Польский С. В. — Москва: ГОУ ВПО МГУЛ, 2008 — 38 c.
	\bibitem{lit7} Закон Ламберта. Модель отражения Фонга. Модель отражения Блинна-Фонга /  [Электронный ресурс] // Компьютерная графика: теория и алгоритмы : [сайт]. — URL: \url{https://compgraphics.info/3D/lighting/phong_reflection_model.php} (дата обращения: 18.09.2024).
	\bibitem{lit8} Мальцев А. В., Михайлюк М. В. Моделирование теней в виртуальных сценах с направленными источниками освещения [Текст] / Мальцев А. В., Михайлюк М. В. // Информационные технологии и вычислительные системы. — 2010. — № 2. — С. 6.
	\bibitem{lit9}  Документация по языку $C\#$ /  [Электронный ресурс] // Microsoft : [сайт]. — URL: \url{https://learn.microsoft.com/ru-ru/dotnet/csharp/} (дата обращения: 08.07.2024).
	\bibitem{lit10}  Класс Stopwatch /  [Электронный ресурс] // Microsoft : [сайт]. — URL: https://learn.microsoft.com/ru-ru/dotnet/csharp/ (дата обращения: 08.12.2024).
	\bibitem{lit11}  Исходный код программы /  [Электронный ресурс] // GitHub : [сайт]. — URL: \url{https://github.com/Kori-Tamashi/bmstu/tree/course_work_cg/course_works/computer_graphics/code}
	\bibitem{lit12} Адамс Д. Программирование RPG игр с использованием DirectX [Текст] / Адамс Д. — 2-е изд. — : Thomson Course Technology PTR, 2004 — 811 c.
	\bibitem{lit13} Порт вывода и система координат / [Текст] // Учебное пособие <<Компьютерная визуализация>>. — Санкт-Петербург: ИТМО, 2003. — С. 23.
	\bibitem{lit14}  Матрицы камеры /  [Электронный ресурс] // UnigineDev : [сайт]. — URL: \url{https://developer.unigine.com/ru/docs/latest/code/fundamentals/matrices/index} (дата обращения: 23.10.2024).
	\bibitem{lit15} Порев В. Компьютерная графика [Текст] / Порев В. — 1-е изд. — Санкт-Петербург: БХВ-Петербург, 2002 — 432 c.
	\bibitem{lit16}  Документация по OpenGL /  [Электронный ресурс] // Open.GL API Documentation : [сайт]. — URL: \url{https://docs.gl/} (дата обращения: 08.12.2024).
	\bibitem{lit17}  DirectX 11 API Documentation /  [Электронный ресурс] // NVIDIA TXAA 3.0 RC1 Documentation : [сайт]. — URL: \url{https://docs.nvidia.com/gameworks/content/gameworkslibrary/postworks/dx11_api.html} (дата обращения: 08.12.2024).
	
\end{thebibliography}
