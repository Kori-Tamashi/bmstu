\ssr{ЗАКЛЮЧЕНИЕ}

В ходе выполнения данной курсовой работы была успешно достигнута поставленная цель -- разработка программы визуализации композиции трехмерных многогранных примитивов. Разработанная программа не только демонстрирует возможности визуализации трехмерных объектов, но и предоставляет пользователю гибкие настройки для работы с их геометрическими и оптическими параметрами.

В процессе разработки были решены все обозначенные задачи:
\begin{itemize}[label=--]
	\item описаны объекты сцены, что позволило создать четкое представление о визуализируемых элементах;
	\item проанализированы существующие алгоритмы компьютерной графики для генерации реалистичных моделей и трехмерной сцены, что обеспечило выбор наиболее эффективных решений;
	\item выбраны наиболее подходящие алгоритмы для достижения поставленной цели, что способствовало созданию качественной визуализации;
	\item спроектированы архитектура и графический интерфейс приложения, что обеспечило удобство взаимодействия пользователя с программой;
	\item выбраны средства реализации программного обеспечения, что позволило эффективно использовать доступные ресурсы и технологии;
	\item реализованы выбранные алгоритмы и структуры данных;
	\item проведено исследование быстродействия разработанного приложения.
\end{itemize}

Работа позволила глубже понять основные принципы и алгоритмы, используемые в компьютерной графике, а также их применение для создания реалистичных трехмерных сцен.

Эффективность программы можно улучшить заменой программной реализации низкоуровневых этапов рендеринга сцены на реализацию поддерживаемую аппаратно из интерфейсов графических движков таких как \textit{OpenGL}~\cite{lit16} и \textit{DirectX}~\cite{lit17}.